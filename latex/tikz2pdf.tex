\documentclass[tikz, table]{standalone}
% \usepackage[pdftex,active,tightpage]{preview}
\usepackage[active,tightpage]{preview}

% Beginning of common specifications %%%%

% \usepackage[utf8]{inputenc}
% \usepackage[utf8]

\usepackage[skins]{tcolorbox}
\usepackage{fontspec}
\usepackage{pgfplots}
\pgfplotsset{compat=1.15}
\usepackage{
tikz,
enumitem,
xcolor,
colortbl,
% fontspec,
ulem,
varwidth,
graphicx}

\usetikzlibrary{
    shapes.geometric,
    % shapes.symbols,
    shapes,
    arrows,
    arrows.meta,
    positioning,
    fit,
    backgrounds,
    calc,
    decorations,
    % calligraphy,
    decorations.pathreplacing,
    % decorations.pathmorphing,
    decorations.fractals,
    lindenmayersystems,
    matrix,
    intersections,
    shapes.multipart
}

% defining some common colours for the shapes to be created %%%%
% \definecolor{McBoxColor}{rgb}{0.13, 0.55, 0.13}
% \definecolor{McBoxColor}{rgb}{0.33, 0.42, 0.18}
\definecolor{McBoxColor}{rgb}{0.01, 0.75, 0.24}
% \definecolor{BnsCropBoxColor}{rgb}{0.55, 0.71, 0.0}
\definecolor{BnsCropBoxColor}{rgb}{0.01, 0.75, 0.24}
% \definecolor{graphicbackground}{rgb}{0.96,0.96,0.8}
\definecolor{graphicbackground}{rgb}{0.98, 0.98, 0.82}

\definecolor{ConstraintsColor}{rgb}{0.63, 0.63, 0.0}
\definecolor{YieldColor}{rgb}{0.93, 0.92, 0.74}

%\definecolor{BnsDomainColor}{rgb}{0.98, 0.94, 0.9}
\definecolor{BnsDomainColor}{rgb}{0.99, 0.96, 0.9}

% \definecolor{McDomainColor}{rgb}{0.74, 0.83, 0.9}
\definecolor{McDomainColor}{rgb}{0.9, 0.9, 0.98}

% \definecolor{ModellingInterfaceDomainColor}{rgb}{0.94, 1.0, 0.94}
\definecolor{ModellingInterfaceDomainColor}{rgb}{0.83, 0.83, 0.83}

% \definecolor {RedBnsBulletsColor} {rgb}{1, .5, 0}
% \definecolor{RedBnsBulletsColor}{rgb}{1.0, 0.6, 0.4}
% \definecolor{RedBnsBulletsColor}{rgb}{0.43, 0.21, 0.1}
\definecolor{RedBnsBulletsColor}{rgb}{1.0, 0.13, 0.32}
\definecolor{BlackBnsBulletsColor}{rgb}{0.0, 0.0, 0.0}

% \definecolor{ModellingInterfaceBulletsColor}{rgb}{0.63, 0.84, 0.71} This not currently needed

% \definecolor{BlueMcBulletsColor}{rgb}{0.28, 0.57, 0.81}
\definecolor{BlueMcBulletsColor}{rgb}{0.0, 0.0, 1.0}
\definecolor{BlackMcBulletsColor}{rgb}{0.0, 0.0, 0.0}

\definecolor{LitleNodeFill}{rgb}{0.97, 0.97, 1.0}

 \definecolor{BarColor}{rgb}{0.56, 0.93, 0.56}
 \definecolor{BarFill}{rgb}{0.56, 0.93, 0.56}


% Rised text 

% Bullet styles definition %%%%
% this is taken from https://tex.stackexchange.com/questions/329990/how-do-i-change-the-color-of-itemize-bullet-specific-and-default

\newlist{RedBnsBullets}{itemize}{1}
\setlist[RedBnsBullets]{
label=\textcolor{RedBnsBulletsColor}{\textbullet},
leftmargin=*,
topsep=0ex,
partopsep=0ex,
parsep=0ex,
itemsep=0ex,
% font=\normalfont\bfseries\color{RedBnsBulletsColor},
before={\color{RedBnsBulletsColor}\itshape}}

\newlist{GrayModellingInterfaceBullets}{itemize}{1}
\setlist[GrayModellingInterfaceBullets]{
label=\textcolor{gray}{\textbullet},
leftmargin=*,
topsep=0ex,
partopsep=0ex,
parsep=0ex,
itemsep=0ex,
% font=\normalfont\bfseries\color{RedBnsBulletsColor},
before={\color{gray}\itshape}}

\newlist{BlackBnsBullets}{itemize}{1}
\setlist[BlackBnsBullets]{
label=\textcolor{BlackBnsBulletsColor}{\textbullet},
leftmargin=*,
topsep=0ex,
partopsep=0ex,
parsep=0ex,
itemsep=0ex,
% font=\normalfont\bfseries\color{RedBnsBulletsColor},
before={\color{BlackBnsBulletsColor}\itshape}}

\newlist{BlueMcBullets}{itemize}{1}
\setlist[BlueMcBullets]{
label=\textcolor{BlueMcBulletsColor}{\textbullet},
leftmargin=*,
topsep=0ex,
partopsep=0ex,
parsep=0ex,
itemsep=0ex,
% font=\normalfont\bfseries\color{BlueMcBulletsColor},
before={\color{BlueMcBulletsColor}\itshape}}

\newlist{BlackMcBullets}{itemize}{1}
\setlist[BlackMcBullets]{
label=\textcolor{BlackMcBulletsColor}{\textbullet},
leftmargin=*,
topsep=0ex,
partopsep=0ex,
parsep=0ex,
itemsep=0ex,
% font=\normalfont\bfseries\color{BlueMcBulletsColor},
before={\color{BlackMcBulletsColor}\itshape}}


% Defining a common font %%%%

% \setsansfont{Arial}
% \setmonofont[Color={0019D4}]{Courier New}
\setromanfont{Times New Roman}

% Defining the basic shape components for the flowchart %%%%

%% Defining common variables for node dimensions %%%%

\pgfmathsetmacro{\MinNodeWidth}{1cm}
\pgfmathsetmacro{\MinNodeHigth}{1cm}

\pgfmathsetmacro{\MinSubNodeWidth}{1cm}
\pgfmathsetmacro{\MinSubNodeHigth}{1cm}

\pgfmathsetmacro{\NodeTextWidth}{5cm}


%%%%%%%%%%%%%%%%%%%%%%%%%%%
\begin{document}


%% Styles specifications %%%%

%%%% Main containers specifications %%%%

\tikzstyle{every node}=[font=\Large]

\tikzstyle{BnsBox} = [rectangle, outer sep =0pt, inner sep=5pt, minimum width=\MinNodeWidth, minimum height=\MinNodeHigth, draw=BnsDomainColor, fill=BnsDomainColor]

\tikzstyle{ModellingInterfaceBox} = [rectangle, outer sep =0pt, inner sep=5pt, minimum width=\MinNodeWidth-3cm, minimum height=\MinNodeHigth,draw=ModellingInterfaceDomainColor, fill=ModellingInterfaceDomainColor]

\tikzstyle{McBox} = [rectangle, outer sep =0pt, inner sep=5pt, minimum width=\MinNodeWidth, minimum height=\MinNodeHigth,draw=McDomainColor, fill=McDomainColor]

%%%% arrows styles %%%%

% \tikzstyle{thickarrow} = [thick,->,>=stealth, very thick]

\tikzstyle{thickarrow} = [thick,-{Stealth[scale=1.3,angle'=45,open]},very thick]

\tikzstyle{dashedarrow} = [dashed,-{Stealth[scale=1.3,angle'=45,open]},very thick]

\tikzstyle{thickes} = [thick,very thick]

\tikzstyle{dashes} = [dashed,very thick]

\tikzstyle {BnsStyle} = [
% rounded rectangle,
ellipse,
% sep = 0pt
% ysep = 0pt,
% xsep = 0pt,
thick,
inner sep=0pt,
minimum width=\MinNodeWidth,
minimum height=\MinNodeHigth,
text width=\NodeTextWidth,
draw=white,
fill=white,
% font=\sffamily,
text centered, 
align=left
% node distance=5mm,
% text height=1.5ex,
% text depth=.25ex
]

\tikzstyle {CloudyBnsStyle} = [
cloud, 
cloud puffs=30,
cloud ignores aspect,
% cloud puff arc=120,
draw=McDomainColor,
fill=McDomainColor,
thick,
inner sep=0pt,
% node distance=2cm and 4cm,
% minimum width=\MinNodeWidth,
% minimum height=\MinNodeHigth,
% minimum width=3cm, 
% minimum height=1cm,
% text width=\NodeTextWidth,
text centered, 
align=center
]

\tikzstyle {FlexBnsStyle} = [
% rounded rectangle,
ellipse,
% sep = 0pt
% ysep = 0pt,
% xsep = 0pt,
draw=McDomainColor,
fill=McDomainColor,
thick,
inner sep=0pt,
% node distance=2cm and 4cm,
% minimum width=\MinNodeWidth,
% minimum height=\MinNodeHigth,
% text width=\NodeTextWidth,
% font=\sffamily,
text centered, 
align=center
% node distance=5mm,
% text height=1.5ex,
% text depth=.25ex
]

\tikzstyle {McStyle} = [
% UpCutRectangle, 
chamfered rectangle,
chamfered rectangle corners=north east,
inner sep=0pt,
minimum width=\MinNodeWidth,
minimum height=\MinNodeHigth,
text width=\NodeTextWidth,
draw=white, 
fill=white,
% font=\sffamily,
text centered, 
align=left
% node distance=5mm,
% text height=1.5ex,
% text depth=.25ex
]

\tikzstyle {FlexMcStyle} = [
% ultra thick,
line width = 1.5mm,
% UpCutRectangle, 
chamfered rectangle,
chamfered rectangle corners=north east,
inner sep=0pt,
% minimum width=\MinNodeWidth,
% minimum height=\MinNodeHigth,
% text width=\NodeTextWidth,
draw=McDomainColor, 
fill=McDomainColor,
% font=\sffamily,
text centered, 
align=center
% node distance=5mm,
% text height=1.5ex,
% text depth=.25ex
]
\tikzstyle{McStyleSub} = [
BlackMcBulletsColor,
rectangle, 
double distance=2pt,
thin,
inner sep=2pt,
minimum width=\MinNodeWidth,
minimum height=\MinNodeHigth,
text width=\NodeTextWidth,
draw=BnsDomainColor, 
fill=white,
% font=\sffamily,
text centered, 
align=left
% node distance=5mm,
% text height=1.5ex,
% text depth=.25ex
]

\tikzstyle{FitnodesStyle} = [
ultra thick,
rectangle, 
inner sep=0pt,
minimum width=\MinNodeWidth,
minimum height=\MinNodeHigth,
text width=\NodeTextWidth,
draw=white, 
fill=white,
% font=\sffamily
% text centered, 
% align=left
% node distance=5mm,
% text height=1.5ex,
% text depth=.25ex
]

\tikzstyle{BraceStyle} = [
white,
decorate, 
line width=1mm, 
decoration={brace,amplitude=10pt}]

\tikzstyle {ModelsStyle} = [
shape = star, 
star points=3, 
star point ratio=1, 
% star points=3,
% star point ratio=1,
minimum size=3cm,
inner sep=0pt,
outer sep=0pt,
% minimum width=0pt,
% minimum height=0pt,
% text width=\NodeTextWidth,
draw=black, 
fill=white,
% font=\sffamily,
text centered, 
align=center
% node distance=5mm,
% text height=1.5ex,
% text depth=.25ex
]

\tikzstyle {TransitionalNodesStyle} = [
rectangle, 
inner sep=0pt,
minimum width=\MinNodeWidth,
minimum height=\MinNodeHigth,
text width=10cm,
draw=white, 
fill=white,
opacity = 0.8,
% xshift = 1mm,
% font=\sffamily,
% font=\normalfont\underline\color{black},
% before={\color{black}\itshape},
text centered, 
align=center
% node distance=5mm,
% text height=1.5ex,
% text depth=.25ex
]

\tikzstyle {ApproachNodesStyle} = [
rectangle, 
% inner sep=0pt,
minimum width=11cm,
minimum height=\MinNodeHigth,
% text width=\NodeTextWidth,
draw=white, 
fill=white,
% font=\sffamily,
text centered, 
align=center
% node distance=5mm,
% text height=1.5ex,
% text depth=.25ex
]

\tikzstyle {ApproachTitlesStyle} = [
rounded rectangle,
% ellipse,
% sep = 0pt
% ysep = 0pt,
% xsep = 0pt,
thick,
inner sep=0pt,
minimum width=\MinNodeWidth,
minimum height=\MinNodeHigth,
text width=\NodeTextWidth,
draw=white,
fill=white,
% font=\sffamily,
text centered, 
align=center
% node distance=5mm,
% text height=1.5ex,
% text depth=.25ex
]

\tikzstyle {BnsBarplotStyle} = [
rectangle split, 
rectangle split parts=2,
% sep = 0pt
% ysep = 0pt,
% xsep = 0pt,
thick,
inner sep=0pt,
minimum width=\MinNodeWidth,
minimum height=\MinNodeHigth,
% text width=\NodeTextWidth,
draw=McDomainColor
% fill=white,
% rectangle split part fill={BnsDomainColor, white}
% font=\sffamily,
% text centered, 
% align=center
% rectangle split part align={center,center}
% node distance=5mm,
% text height=1.5ex,
% text depth=.25ex
% every text node part/.style={text=black},
% every lower node part/.style={text=blue}
]

\tikzstyle {BnsMultipartStyle} = [
rectangle split, 
rectangle split parts=2,
thick,
inner sep=2pt,
% outer sep =0pt,
minimum width=\MinNodeWidth,
minimum height=\MinNodeHigth,
draw=white,
rectangle split part fill={graphicbackground, white},
every one node part/.style={font=\LARGE\bf}
]

\tikzstyle {McMultipartStyle} = [
rectangle split, 
rectangle split parts=2,
thick,
inner sep=0pt,
minimum width=\MinNodeWidth,
minimum height=\MinNodeHigth,
draw=white,
rectangle split part fill={graphicbackground, white}
]

\tikzstyle {BnsBarplotTitleStyle} = [
% UpCutRectangle, 
rectangle,
inner sep=0pt,
minimum width=\MinNodeWidth,
minimum height=\MinNodeHigth,
%text width=\NodeTextWidth,
draw=graphicbackground, 
fill=graphicbackground,
% font=\sffamily,
font=\LARGE\bf,
text centered, 
align=left
% node distance=5mm,
% text height=1.5ex,
% text depth=.25ex
]

% Preamble: \pgfplotsset{width=7cm,compat=1.3}
\pgfplotsset{AxisStyle/.style={
                    y=20pt,
                    enlarge y limits={abs=10pt},
                    xbar,
                    /pgf/bar width=15pt,% bar width in barplot
                    xmin=0,
                    % xmax=1,
                    y axis line style = {opacity = 0},
                    axis x line       = none,
                    tickwidth         = 0pt,
                    ytick             = data,
                    nodes near coords, nodes near coords align={horizontal},
                    scaled x ticks=false,
                    every node near coord/.style={
                         /pgf/number format/fixed,
                         /pgf/number format/precision=3}
                    }
}

\pgfplotsset{AddPlotStyle/.style={ultra thin,
color=black,
% fill=ModellingInterfaceDomainColor, 
fill=white,
every node near coord/.style={text=black}
}
}

% background color definition from pgfmanual-en-macros.tex
% key to change color
\pgfkeys{/tikz/.cd,
  background color/.initial=BnsDomainColor,
  background color/.get=\backcol,
  background color/.store in=\backcol,
}
\tikzset{background rectangle/.style={
    fill=\backcol,
  },
  use background/.style={    
    show background rectangle
  }
}

\newcommand{\specialcell}[2][c]{%
  \begin{tabular}[#1]{@{}c@{}}#2\end{tabular}}

% End of common specifications %%%%
\begin{preview}
%% TIKZ_CODE %%
\end{preview}

\end{document}